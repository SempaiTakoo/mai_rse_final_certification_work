\chapter{Данные}
\label{ch:data}

    Определение пожаров на видео, снятых с камер дронов, является важной задачей для предотвращения распространения огня и минимизации ущерба. Для этого необходимы специализированные нейронные сети, которые могут автоматически идентифицировать признаки пожара на видеоматериалах. Ключевым этапом в создании таких нейронных сетей является сбор и подготовка данных для их обучения. Данные для обучения нейронной сети, предназначенной для определения пожаров на видео с камер дронов, должны удовлетворять ряду условий, чтобы обеспечить высокую точность и надежность модели.

    Эти условия можно разделить на несколько категорий: качество данных, разнообразие, аннотации, репрезентативность и объём.

    \section{Качество данных}
    \begin{enumerate}
        \item Разрешение видео: Видео должны быть по качеству приближены к видео, снятым на камеру дрона. 
        \item Четкость и контрастность: Кадры должны быть четкими и иметь хороший контраст, чтобы огонь и дым были легко различимы.
        \item Минимум артефактов и шума: Видео должны быть очищены от артефактов и шума, которые могут затруднить обучение модели. Это может включать устранение искажений, вызванных плохими погодными условиями или низким качеством съемки.
    \end{enumerate}

    \section{Разнообразие данных}
    \begin{enumerate}
        \item Разные типы пожаров: Видео должны включать различные типы пожаров, такие как лесные пожары, промышленные пожары, пожары в жилых зданиях и т.д.
        \item Разные условия съемки: Данные должны включать съемку в различных условиях, включая дневное и ночное время, различные погодные условия (дождь, снег, туман), а также разные сезоны года.
        \item Разные ракурсы и высоты съемки: Видео должны быть сняты с различных высот и углов, чтобы модель могла научиться определять пожары независимо от положения дрона.
        \item Фоновые условия: Данные должны включать разные фоны (лес, город, промышленная зона), чтобы модель могла различать пожар от других объектов на фоне.
    \end{enumerate}

    \section{Сбор данных}
    Сбор данных является одним из первоочередных этапов в разработке модели обнаружения пожаров на видео с камер дронов. Этот процесс включает в себя собственно сбор видеоматериала, который затем будет использоваться для обучения и тестирования модели. Вот более подробное описание этого этапа:

    \section{Определение источников данных}
    Первый шаг в сборе данных - определение источников, из которых можно получить видеоматериалы. Это может включать в себя следующие источники:
    \begin{itemize}
        \item Открытые базы данных: Существуют различные открытые базы данных, содержащие видеозаписи с дронов. Некоторые из них могут содержать видео с пожарами или другими бедствиями, которые могут быть полезны для обучения модели. Roboflow - это онлайн-платформа для подготовки данных, обучения моделей машинного обучения и развертывания моделей в производственной среде. Она предоставляет инструменты и ресурсы для работы с изображениями и другими данными, необходимыми для обучения нейронных сетей. Одной из ключевых особенностей Roboflow является его простота использования и возможность интеграции в популярные фреймворки машинного обучения, такие как TensorFlow, PyTorch и другие. Платформа предоставляет широкий спектр функций, включая загрузку и аннотирование данных, аугментацию изображений, обучение моделей на основе предварительно обученных архитектур и многое другое. Она также обладает гибкими инструментами для управления данными, позволяя пользователям создавать и организовывать различные проекты, настраивать процессы аннотирования и аугментации данных, а также мониторить процесс обучения моделей. Roboflow предлагает различные планы, включая бесплатный план для небольших проектов и платные планы с расширенными возможностями для коммерческих и профессиональных пользователей. Платформа также предоставляет облачное хранилище для данных, что упрощает их управление и доступность для команды проекта. Основная цель Roboflow - сделать процесс подготовки данных и обучения моделей машинного обучения более доступным и эффективным для разработчиков и исследователей. Его интуитивно понятный интерфейс и богатый набор функций делают его популярным выбором для проектов компьютерного зрения и других областей машинного обучения.
        \item Собственные съемки: В случае отсутствия подходящих данных в открытых источниках, можно провести собственные съемки с помощью камер дронов. Это может быть особенно полезно для получения данных в специфических условиях или местах.
    \end{itemize}

    \section{Составление набора данных}
    После определения источников данных необходимо собрать сам набор данных. Это включает в себя следующие шаги:
    \begin{itemize}
        \item Выбор видеоматериалов: Из доступных источников выбираются видеозаписи, которые наилучшим образом соответствуют целям и требованиям задачи. В случае обнаружения пожаров важно, чтобы видео содержали различные типы пожаров и сценарии, чтобы модель обучалась на разнообразных данных.
        \item Формирование метаданных: Для каждого видео создаются метаданные, которые описывают его содержание. Это может включать в себя информацию о дате и месте съемки, типе пожара, погодных условиях и другие сведения, которые могут быть полезны для последующей обработки и анализа.
    \end{itemize}

    \section{Аннотирование данных}
    Для обучения модели необходимо аннотировать каждое видео с указанием местоположения пожара на кадрах. Это делается путем создания ограничивающих рамок (bounding boxes) вокруг областей с пожаром на каждом кадре видео. Аннотации могут быть созданы вручную с помощью специальных инструментов для аннотации данных или с использованием алгоритмов компьютерного зрения.

    \section{Оценка качества данных}
    После сбора и аннотирования данных необходимо провести оценку их качества. Это включает в себя проверку наличия ошибок в аннотациях, а также оценку полноты и разнообразия данных. Если обнаружены ошибки или недостатки, необходимо их исправить или дополнить набор данных соответствующим образом.

    \section{Создание разделения данных}
    Наконец, важно разделить набор данных на тренировочный, валидационный и тестовый наборы. Это позволяет оценить производительность модели на независимом наборе данных и предотвратить переобучение. Обычно используется пропорция около 70-80% для тренировочного набора, 10-15% для валидационного и 10-15% для тестового.

    \section{Архитектура модели}
    При обнаружении пожаров на видео с камер дронов используются различные модели и архитектуры глубокого обучения. Давайте рассмотрим несколько из них:
    \begin{enumerate}
        \item YOLO (You Only Look Once) \\
        Основные характеристики:
            \begin{itemize}
                \item Единоразовое предсказание: YOLO использует одинаковую сеть для предсказания координат ограничивающих рамок и вероятностей классов объектов.
                \item Детекция в реальном времени: YOLO способен работать в реальном времени, благодаря чему его широко используют для видеонаблюдения и дронов.
                \item Эффективность: YOLO обладает высокой скоростью обработки и высокой точностью даже на сложных видеопотоках.
            \end{itemize}
        \item SSD (Single Shot MultiBox Detector) \\
        Основные характеристики:
            \begin{itemize}
                \item Многомасштабные признаки: SSD использует множество признаков различных масштабов для обнаружения объектов разного размера.
                \item Детекция в один шаг: Архитектура SSD обладает высокой скоростью обработки за счет выполнения детекции объектов в один шаг.
                \item Высокая точность: SSD способен обнаруживать объекты с высокой точностью благодаря использованию множества признаков.
            \end{itemize}
        \item Faster R-CNN (Faster Region-based Convolutional Neural Network) \\
        Основные характеристики:
            \begin{itemize}
                \item Региональные сверточные сети: Faster R-CNN использует региональные сверточные сети для предложения областей, где могут находиться объекты.
                \item Сеть для обнаружения объектов: После предложения областей, Faster R-CNN использует сверточную сеть для классификации и точной локализации объектов.
                \item Более сложная архитектура: Faster R-CNN имеет более сложную архитектуру по сравнению с YOLO и SSD, что может привести к более высокой точности, но меньшей скорости обработки.
            \end{itemize}
        \item RetinaNet \\
        Основные характеристики:
            \begin{itemize}
                \item Простота и эффективность: RetinaNet представляет собой простую и эффективную архитектуру, которая способна обнаруживать объекты разного размера.
                \item Фокус на объектах разного размера: RetinaNet использует фокусированный подход к обнаружению объектов разного размера, что делает его эффективным для обнаружения пожаров на видео с камер дронов.
                \item Использование Focal Loss: В отличие от других архитектур, RetinaNet использует Focal Loss для борьбы с проблемой несбалансированных классов.
            \end{itemize}
    \end{enumerate}

    Нами была выбрана модель YOLO по ряду нескольких причин. YOLO - это инновационная архитектура нейронных сетей, разработанная для задачи обнаружения объектов в изображениях и видео. Основное отличие YOLO заключается в том, что она осуществляет обнаружение объектов всего один раз для всего изображения в одном прямом проходе через нейронную сеть. Это позволяет модели предсказывать ограничивающие рамки (bounding boxes) с классами объектов и их вероятностями в реальном времени, делая YOLO идеальным выбором для приложений, требующих высокой скорости обработки, таких как видеонаблюдение и автономные автомобили.

    YOLO применяет концепцию "конец в конец" (end-to-end), принимая на вход изображение и возвращая ограничивающие рамки с классами объектов, обеспечивая тем самым простоту и эффективность архитектуры. Одна из ключевых особенностей YOLO - это использование множества масштабов для обнаружения объектов различных размеров на изображении. Благодаря этому, модель способна обнаруживать как крупные, так и мелкие объекты, обеспечивая высокую точность и полноту обнаружения.

    YOLO является расширяемым и может быть адаптирован для различных задач обнаружения объектов, включая обнаружение лиц, транспортных средств, пожаров и многих других объектов. Последующие версии YOLO, такие как YOLOv3 и YOLOv4, вносят дополнительные улучшения в архитектуру и производительность, делая YOLO ведущим выбором для широкого спектра приложений в области компьютерного зрения.

    YOLO выделяется среди других моделей детекции объектов благодаря нескольким ключевым преимуществам. Давайте рассмотрим, почему YOLO может считаться лучшим выбором для задач, таких как определение пожара на видео с камеры дрона.

    \section{Преимущества YOLO}
    \begin{enumerate}
        \item Высокая скорость \\
        YOLO оптимизирован для скорости, что позволяет ему обрабатывать видео в реальном времени. Это критически важно для задач обнаружения пожаров, где необходимо быстрое реагирование:
            \begin{itemize}
                \item ●	Единая архитектура: YOLO выполняет детекцию в один этап, в отличие от других моделей, таких как R-CNN, которые используют несколько этапов (например, сначала выделяют регионы интереса, а затем классифицируют их).
                \item ●	Эффективное использование GPU: YOLO разработан с учетом эффективного использования вычислительных ресурсов GPU, что позволяет добиться высокой производительности.
            \end{itemize}
        \item Консистентность \\
        YOLO имеет высокую точность при детекции объектов:
            \begin{itemize}
                \item ●	Глобальная контекстная информация: YOLO рассматривает изображение целиком при обучении и предсказании, что помогает модели учитывать контекст и уменьшать количество ложных срабатываний.
                \item ●	Сбалансированная точность и полнота: YOLO эффективно балансирует между высокой точностью (precision) и полнотой (recall), что критично для задач, где важно не пропустить ни одного пожара и минимизировать ложные тревоги.
            \end{itemize}
        \item Простота и удобство использования \\
        YOLO имеет удобную и относительно простую архитектуру, которая позволяет легко адаптировать модель под конкретные задачи:
            \begin{itemize}
                \item ●	Настройка и обучение: YOLO проще в настройке и обучении по сравнению с более сложными архитектурами, такими как Faster R-CNN.
                \item ●	Поддержка и документация: YOLO имеет широкую поддержку со стороны сообщества и хорошую документацию, что облегчает его использование и адаптацию для различных приложений.
            \end{itemize}
        \item Универсальность \\
        YOLO можно применять для детекции множества типов объектов, и она хорошо справляется с различными задачами:
            \begin{itemize}
                \item ●	Множественные классы объектов: YOLO поддерживает детекцию различных классов объектов одновременно, что может быть полезно в сценариях, где необходимо выявлять не только пожары, но и другие объекты, такие как люди или транспортные средства.
                \item ●	Адаптация к новым условиям: Благодаря своей архитектуре, YOLO можно адаптировать к новым типам данных и условиям съемки, что делает его гибким инструментом.
            \end{itemize}
    \end{enumerate}

    \section{Обучение модели}
    Процесс обучения модели YOLOv8 для обнаружения пожаров на видео с камер дронов включает в себя ряд шагов, которые необходимо выполнить. Вот подробное описание каждого шага:

    \begin{enumerate}
        \item \textbf{Подготовка окружения и инструментов}
            \begin{itemize}
                \item Установите необходимые библиотеки и фреймворки, такие как PyTorch, OpenCV, и другие зависимости.
                \item Подготовьте вычислительные ресурсы, включая GPU, для ускорения процесса обучения.
            \end{itemize}
        \item \textbf{Загрузка предобработанных данных}
            \begin{itemize}
                \item Загрузите предварительно подготовленные данные, включая изображения и соответствующие аннотации с помощью специальных инструментов для работы с YOLO форматом.
                \item Разделите данные на тренировочный, валидационный и тестовый наборы, учитывая разнообразие классов объектов.
            \end{itemize}
        \item \textbf{Подготовка конфигурации модели}
            \begin{itemize}
                \item Скачайте предварительно обученные веса для модели YOLOv8, чтобы использовать их в качестве начальных значений.
                \item Создайте файл конфигурации, в котором указываются параметры модели, такие как количество классов, пути к данным и весам, размеры изображений и т.д.
            \end{itemize}
        \item \textbf{Обучение модели}
            \begin{itemize}
                \item Загрузите предварительно обученную модель YOLOv8 с помощью PyTorch.
                \item Настройте параметры обучения, такие как скорость обучения, коэффициенты регуляризации и количество эпох.
                \item Запустите процесс обучения модели на тренировочном наборе данных.
                \item Оценивайте процесс обучения по мере продвижения, анализируя метрики потерь, точности и другие показатели.
            \end{itemize}
        \item \textbf{Оценка модели}
            \begin{itemize}
                \item Оцените производительность обученной модели на валидационном наборе данных.
                \item Используйте метрики оценки, такие как точность, полнота, F1-мера, для оценки качества модели.
                \item Визуализируйте результаты обнаружения, чтобы понять, как модель справляется с задачей.
            \end{itemize}
        \item \textbf{Тонкая настройка и оптимизация}
            \begin{itemize}
                \item Проанализируйте результаты и выявите возможные улучшения.
                \item Оптимизируйте параметры модели и процесс обучения для достижения лучших результатов.
                \item При необходимости повторите процесс обучения с учетом полученных результатов для дальнейшего улучшения модели.
            \end{itemize}
        \item \textbf{Тестирование модели}
            \begin{itemize}
                \item Протестируйте обученную модель на тестовом наборе данных для оценки ее обобщающей способности.
                \item Проведите дополнительные эксперименты, чтобы оценить производительность модели в реальных условиях.
            \end{itemize}
    \end{enumerate}

    В итоге, процесс обучения модели YOLOv8 для обнаружения пожаров на видео с камер дронов требует тщательной подготовки данных, настройки параметров модели и систематического анализа результатов обучения.

    Оценка модели - это важный этап в процессе разработки алгоритма обнаружения пожаров на видео с камер дронов. Оценка позволяет оценить качество модели, ее способность обобщения на новые данные и выявить ее сильные и слабые стороны. Вот подробное описание процесса оценки модели:
    \begin{enumerate}
        \item Разделение данных \\
        Перед началом оценки модели данные обычно разделяют на несколько наборов: тренировочный, валидационный и тестовый. Тренировочный набор используется для обучения модели, валидационный — для выбора гиперпараметров и контроля переобучения, а тестовый — для окончательной оценки производительности модели на независимых данных.
        
        \item Выбор метрик оценки \\
        Определите метрики, которые будут использоваться для оценки производительности модели. В контексте обнаружения пожаров на видео с камер дронов, такие метрики могут включать в себя:
        \begin{itemize}
            \item Точность (Accuracy): Метрика точности (Accuracy) является одной из основных метрик в оценке производительности моделей машинного обучения. Она измеряет долю правильно классифицированных примеров среди всех примеров в наборе данных. Точность позволяет оценить, насколько хорошо модель классифицирует объекты верно, и предоставляет общую оценку её способности делать верные предсказания. Для вычисления точности необходимо разделить количество правильно классифицированных примеров на общее количество примеров в наборе данных. Например, если у нас есть 100 изображений и модель правильно классифицирует 90 из них, то точность будет равна 90\%. Точность является простой и интерпретируемой метрикой, которая часто используется для оценки моделей в различных областях, таких как классификация текста, изображений или аудио. Однако стоит помнить, что точность не всегда является единственно правильной метрикой для оценки модели. Она может быть непоказательной в случае несбалансированных классов или неправильной обработки ошибок первого и второго рода. Поэтому при анализе точности важно учитывать контекст задачи и дополнительно оценивать другие метрики.
            
            \item Полнота (Recall): измеряет долю обнаруженных положительных примеров относительно всех реально существующих положительных примеров в наборе данных. Она позволяет оценить способность модели обнаруживать все реальные положительные примеры без пропусков. Полнота выражается как отношение числа верно классифицированных положительных примеров к общему числу положительных примеров. Для вычисления полноты необходимо разделить количество верно классифицированных положительных примеров на сумму верно классифицированных положительных примеров и ложно отрицательных примеров (примеров, которые модель неправильно определила как негативные). Например, если в наборе данных есть 100 положительных примеров, и модель правильно обнаружила 90 из них, а 10 не обнаружила, то полнота будет равна 90\%. Полнота важна, когда стоит задача минимизировать количество пропусков объектов, которые имеют реальное значение (ложноотрицательные результаты). Это особенно важно в задачах, где пропуск объекта может иметь серьёзные последствия, таких как медицинская диагностика или обнаружение аварийных ситуаций. Однако, увеличение полноты может привести к увеличению ложно положительных результатов, поэтому необходимо сбалансировать эту метрику с другими метриками, такими как точность, для достижения оптимального результата.
            
            \item Точность (Precision): представляет собой долю истинно положительных примеров среди всех примеров, которые модель классифицировала как положительные. Она измеряет степень правильности модели в определении объектов положительного класса и позволяет оценить уровень шума в результатах классификации. Точность выражается как отношение числа верно классифицированных положительных примеров к сумме верно классифицированных положительных примеров и ложно положительных примеров (примеров, которые модель неправильно определила как положительные). Например, если модель классифицировала 100 объектов как положительные, из которых 90 действительно являются положительными, а остальные 10 — ложно положительными, то точность составит 90\%. Таким образом, высокое значение точности указывает на то, что модель имеет мало ложно положительных результатов и высокую уверенность в предсказаниях. Точность важна в ситуациях, когда неверное срабатывание (ложно положительный результат) может иметь негативные последствия или дополнительные расходы. Например, в задачах систем безопасности или финансовых расследований. Однако, повышение точности может привести к уменьшению количества обнаруженных объектов (истинно положительных результатов), поэтому необходимо учитывать и балансировать эту метрику с другими метриками, такими как полнота, для достижения оптимального результата в задаче классификации.
            
            \item F1-мера (F1-Score): это гармоническое среднее между точностью и полнотой и используется для оценки производительности моделей машинного обучения в задачах бинарной классификации. Она обеспечивает компромисс между этими двумя метриками, позволяя оценить баланс между количеством истинно положительных и истинно отрицательных результатов при классификации. F1-мера вычисляется как обратное гармоническое среднее точности и полноты. Гармоническое среднее предпочтительно для использования в ситуациях, когда требуется выявить среднее значение между двумя метриками, особенно если они имеют разные диапазоны значений. Это позволяет учесть оба аспекта оценки модели, не учитывая просто среднее арифметическое. Например, если модель имеет высокую точность, но низкую полноту (т.е. много ложно положительных результатов, но мало истинно положительных), или наоборот, F1-мера поможет оценить её общую производительность, учитывая оба аспекта. Высокое значение F1-меры указывает на то, что модель имеет хороший баланс между точностью и полнотой, что является желательным результатом в большинстве задач классификации. F1-мера особенно полезна в ситуациях, когда классы несбалансированы, и когда обе метрики (точность и полнота) имеют равное значение. Однако, стоит помнить, что F1-мера также не учитывает истинно отрицательные результаты (TN), что делает её неидеальным выбором для задач с несбалансированными классами.
        \end{itemize}
        
        \item Оценка на валидационном наборе данных \\
        Запустите обученную модель на валидационном наборе данных и измерьте выбранные метрики оценки. Это поможет определить, насколько хорошо модель обобщает знания на новые данные и предотвратить переобучение.
        
        \item Тонкая настройка и оптимизация \\
        Если результаты на валидационном наборе данных не удовлетворяют требованиям, можно попробовать изменить гиперпараметры модели или архитектуру, чтобы улучшить её производительность. После внесения изменений повторите процесс обучения и оценки на валидационном наборе.
        
        \item Оценка на тестовом наборе данных \\
        Окончательную оценку производительности модели следует проводить на тестовом наборе данных, который не использовался в процессе обучения и валидации. Это позволит оценить обобщающую способность модели и предоставить объективные результаты её работы.
        
        \item Анализ результатов \\
        После завершения оценки модели анализируйте полученные результаты и выявите её сильные и слабые стороны. При необходимости можно провести дополнительные эксперименты или оптимизации для улучшения производительности модели.
        
        \item Документирование и представление результатов \\
        Документируйте все проведённые эксперименты, полученные результаты и основные выводы. Представьте результаты в удобном формате, например, в виде таблиц или графиков, чтобы облегчить их интерпретацию и анализ.
    \end{enumerate}
 
    \section{Анализ результатов}
    \begin{enumerate}
        \item Отчеты и визуализации: Создаются отчеты и визуализации, показывающие результаты работы модели на тестовых данных. Используются графики, диаграммы и примеры кадров с предсказанными объектами.
        \item Обратная связь и корректировки: На основе результатов тестирования и валидации делаются выводы о необходимости дополнительных улучшений модели. Проводятся дополнительные циклы обучения и настройки, если это необходимо.
    \end{enumerate}
    
\endinput