\chapter{Техническая документация}
\label{ch:documentation}

    Эта глава посвящена описанию разработки функционала для Портирования нейронных сетей компьютерного зрения на бортовые вычислительные комплексы БПЛА.  Будут рассмотрены цели, задачи и использованный для них стек технологий.
    
    \section{Цели и задачи}
    \begin{itemize}
        \item Обучить нейросеть
        \item Запустить нейросеть в реальных условиях
    \end{itemize}

    \section{Архитектура и дизайн системы}
    
    \subsection{Общая архитектура}
    Система состоит из следующих основных компонентов:
    \begin{itemize}
        \item Нейросеть: Используется YOLO.
        \item Virtual box: Qemu для запуска нейросети в реальных условиях
    \end{itemize}

    \section{Обучение модели YOLO}
    YOLO (You Only Look Once) — это одна из самых популярных и мощных архитектур для задачи обнаружения объектов в реальном времени. YOLO отличается тем, что выполняет детекцию объектов в одном проходе через сеть, что обеспечивает высокую скорость обработки изображений. Это делает YOLO идеальной для приложений, требующих быстрого и точного обнаружения объектов, таких как системы видеонаблюдения, автономные транспортные средства и роботы.

    \subsection{Подготовка к обучению YOLO}
    
    \subsection{Требования к аппаратному и программному обеспечению}
    Перед началом обучения вам потребуется следующее:
    \begin{itemize}
        \item Графический процессор (GPU): Обучение моделей YOLO требует значительных вычислительных ресурсов. Рекомендуется использовать GPU от NVIDIA (Pascal или более нового поколения) с памятью не менее 8 ГБ.
        \item Фреймворк глубокого обучения: Основные фреймворки для обучения моделей YOLO включают TensorFlow, PyTorch и Keras. Большинство современных реализаций YOLO используют PyTorch благодаря его гибкости и легкости использования.
        \item CUDA и cuDNN: Библиотеки NVIDIA CUDA и cuDNN необходимы для использования GPU.
        \item Программная среда: Установите Python и необходимые библиотеки, такие как NumPy, OpenCV, Matplotlib и другие, в зависимости от конкретной реализации YOLO.
    \end{itemize}

    \subsection{Сбор данных}
    Для обучения модели YOLO необходимо собрать метки данных, включающие следующие компоненты:
    \begin{itemize}
        \item Изображения: Набор изображений, на которых будут проводиться детекции объектов.
        \item Аннотации: Метки, содержащие информацию о положении и категории объектов на изображении.
    \end{itemize}

    \textbf{Подходы к сбору данных включают:}
    \begin{itemize}
        \item Публичные датасеты: Использование существующих датасетов, таких как COCO, Pascal VOC, Open Images и др.
        \item Собственные данные: Сбор и аннотирование собственных данных, что может потребовать значительного времени и усилий.
    \end{itemize}
    
    \subsection{Аннотирование данных}
    Аннотации обычно выполняются вручную с использованием инструментов, таких как LabelImg, CVAT и Vott. Аннотации сохраняются в формате, совместимом с YOLO, например в текстовом формате, где каждый объект представляется строкой: \\
    <Class\-ID> <X\_center> <Y\_center> <Width> <Height>

    \subsection{Форматы данных и преобразования}
    Для использования данных в процессе обучения необходимо преобразовать их в необходимый формат. Большинство реализаций YOLO поддерживают следующие форматы данных:
    \begin{itemize}
        \item Формат YOLO: Стандартный текстовый формат для аннотаций, где каждая строка файла аннотации представляет один объект на изображении.
        \item TFRecord: Формат данных для TensorFlow, который упрощает работу с большими объемами данных.
        \item JSON: Формат данных, часто используемый для хранения аннотаций, таких как COCO.
    \end{itemize}

    \subsection{Конфигурация и обучение модели YOLO}

    \subsection{Настройка архитектуры модели}
    Для настройки модели YOLO необходимо определить несколько ключевых параметров:
    \begin{itemize}
        \item Сеть-направляющая (Backbone): Выбор сети-направляющей, такой как Darknet, EfficientNet, ResNet и др. YOLOv3 и более поздние версии часто используют Darknet.
        \item Размерность сети (Grid size): Определение размеров ячеек сетки (например, 13x13 для YOLOv3), на которые будет разделено изображение.
        \item Якорные метки (Anchors): Определение наборов якорных меток, используемых для предсказания рамок объектов разного масштаба.
    \end{itemize}

    \subsection{Гиперпараметры обучения}
    Основные гиперпараметры, которые необходимо настроить:
    \begin{itemize}
        \item Частота обучения (Learning rate): Скорость изменения весов сети. Обычно используется со стратегиями постепенного уменьшения скорости обучения.
        \item Мини-батч (Mini-batch): Размер мини-батча для обновления градиентов.
        \item Эпохи (Epochs): Количество проходов по всему набору данных.
        \item Аугментация данных (Data augmentation): Применение преобразований данных, таких как повороты, масштабирование и сдвиг по оси, чтобы улучшить обобщающую способность модели.
    \end{itemize}

    \subsection{Запуск процесса обучения}
    Для обучения модели YOLO сначала необходимо настроить конфигурационный файл. 

    Запуск процесса обучения обычно осуществляется с помощью одной из следующих команд (в зависимости от используемого фреймворка):
    \begin{itemize}
        \item PyTorch: \\
          python train.py --data data.yaml --cfg yolov4.yaml --weights '' --batch-size 16
        \item Darknet: \\
          ./darknet detector train data/obj.data cfg/yolov4.cfg yolov4.conv.137
    \end{itemize}

    \subsection{Оценка и улучшение модели}

    \subsection{Метрики оценки}
    Основные метрики, используемые для оценки модели YOLO:
    \begin{itemize}
        \item Точность (Precision): Доля правильно предсказанных положительных объектов среди всех предсказанных положительных объектов.
        \item Полнота (Recall): Доля правильно предсказанных положительных объектов среди все реальных положительных объектов.
        \item Средняя точность (mAP): Среднее значение точности по всем классам объектов.
    \end{itemize}

    \subsection{Техники улучшения модели}

    \begin{itemize}
        \item Предварительное обучение (Transfer Learning): Использование предварительно обученной модели на большом наборе данных и дообучение на вашем наборе данных.
        \item Тонкая настройка гиперпараметров (Hyperparameter Tuning): Настройка гиперпараметров, таких как learning rate, batch size и другие.
        \item Обогащение данных (Data Augmentation): Применение техник аугментации данных для улучшения обобщающей способности модели.
        \item Увеличение базового набора данных: Сбор большего объема данных с разнообразными сценами и объектами.
    \end{itemize}

    \subsection{Развертывание модели YOLO}

    \subsection{Оптимизация модели}
    Для развертывания модели YOLO на устройствах с ограниченными ресурсами, таких как мобильные устройства и встраиваемые системы, могут потребоваться следующие оптимизации:
    \begin{itemize}
        \item Снижение точности (Quantization): Преобразование весов модели с 32-битного float формата в 8-битный integer формат.
        \item Усечение (Pruning): Удаление менее важных весов из модели для уменьшения её размера.
        \item Компиляция в специализированные библиотеки: Использование библиотек, таких как NVIDIA TensorRT, для ускорения инференса.
    \end{itemize}

    \subsection{Развертывание на различных платформах}
    YOLO можно развернуть на следующих платформах:
    \begin{itemize}
        \item Web: Использование TensorFlow.js или ONNX.js для выполнения инференса в браузере.
        \item Мобильные устройства: Использование TensorFlow Lite или Core ML для выполнения инференса на Android и iOS.
        \item Встраиваемые системы: Использование NVIDIA Jetson или Google Coral для инференса на устройствах с ограниченными ресурсами.
    \end{itemize}

    \subsection{Реализация в производственных системах}
    Для успешного развертывания YOLO в производственных системах необходимо учитывать следующие аспекты:
    \begin{itemize}
        \item Мониторинг и управление моделями: Использование инструментов для мониторинга производительности и переразвертывания моделей.
        \item Интеграция с существующими системами
    \end{itemize}

    Обучение моделей YOLO для обнаружения объектов представляет собой сложный, но очень эффективный процесс, который требует тщательной подготовки данных, настройки модели и оптимизации гиперпараметров. С помощью YOLO можно достичь высоких результатов в задачах обнаружения объектов в реальном времени, что делает его идеальным выбором для многих приложений. Правильно организованный процесс обучения и развертывания обеспечит надежную и быструю работу системы обнаружения объектов.

    \section{Запуск нейросети на виртуальной машине}
    QEMU (Quick EMUlator) — это бесплатный и с открытым исходным кодом эмулятор процессоров общего назначения и виртуализатор. Он может эмулировать различные архитектуры процессоров, такие как x86, ARM, PowerPC, SPARC и другие, а также предоставляет полный спектр возможностей для создания виртуализированных окружений. Основные случаи использования QEMU включают тестирование программного обеспечения, выполнение эмуляции аппаратных средств и разработку операционных систем.

    Один из наиболее популярных способов установки QEMU на Linux — использование менеджеров пакетов, таких как APT для Debian/Ubuntu, YUM для CentOS, и Pacman для Arch Linux.

    Debian/Ubuntu:
    \begin{lstlisting}
        sudo apt update
        sudo apt install qemu qemu-kvm libvirt-daemon-system libvirt-clients bridge-utils
    \end{lstlisting}
    
    CentOS:
    \begin{lstlisting}
        sudo yum install qemu-kvm libvirt virt-install bridge-utils
        sudo systemctl start libvirtd
        sudo systemctl enable libvirtd
    \end{lstlisting}
    
    Arch Linux:
    \begin{lstlisting}
        sudo pacman -S qemu libvirt
        sudo systemctl start libvirtd
        sudo systemctl enable libvirtd
    \end{lstlisting}
    
    Windows: \\
    Для установки QEMU на Windows необходимо скачать установочный пакет с официального сайта (https://www.qemu.org/download/ windows) и следовать инструкциям установщика.

    \subsection{Основные компоненты}
    \subsection{Эмуляция процессоров (CPU)}
    QEMU может эмулировать различные процессоры, предоставляя пользователю возможность запускать код, предназначенный для различных архитектур. Примеры поддерживаемых архитектур включают
    \begin{itemize}
        - x86 и x8664
        - ARM
        - MIPS
        - SPARC
        - POWER
    \end{itemize}

    \subsection{Виртуализация}
    QEMU поддерживает KVM (Kernel-based Virtual Machine), что позволяет использовать аппаратную виртуализацию для более высокой производительности. KVM делает QEMU более эффективным, минимизируя оверхед эмуляции кода.

    \subsection{Сетевые возможности}
    QEMU предоставляет различные способы настройки сети, включая пользовательскую модель сети, сетевые мосты (bridging), туннелирование и привязку к хостовому интерфейсу.

    \subsection{Файловая система}
    QEMU поддерживает использование различных типов виртуальных дисков, таких как qcow2, raw, vmdk и другие. Он предоставляет инструменты для управления виртуальными дисками, такие как создание снапшотов и преобразование форматов.

    \subsection{Запуск виртуальной машины}

    \subsection{Базовый запуск}
    Запустить простую виртуальную машину можно с помощью следующей команды:
    \begin{lstlisting}
        qemu-system-x8664 -hda /path/to/diskimage.img -m 1024 -boot d
    \end{lstlisting}
    Где:
    \begin{itemize}
        \item \textit{qemu-system-x8664} — исполняемый файл для эмуляции x86\_64 архитектуры.
        \item \textit{-hda /path/to/diskimage.img} — путь к образу жесткого диска.
        \item \textit{-m 1024} — выделение 1024 МБ оперативной памяти.
        \item \textit{-boot d} — установка последовательности загрузки (d-устройство загрузки CDROM/DVD).
    \end{itemize}

    \subsection{Использование KVM}
    Для увеличения производительности можно использовать KVM:
    \begin{lstlisting}
        qemu-system-x8664 -enable-kvm -hda /path/to/diskimage.img -m 1024 -boot d
    \end{lstlisting}

    \subsection{Настройка сети}
    Для настройки сетевого интерфейса с NAT (Network Address Translation) можно использовать следующие параметры:
    \begin{lstlisting}
        qemu-system-x8664 -hda /path/to/diskimage.img -m 1024 -boot d -netdev user,id=net0 -device e1000,netdev=net0
    \end{lstlisting}

    \subsection{Проброс USB-устройств}
    Чтобы пробросить USB-устройство в виртуальную машину, можно использовать:
    \begin{lstlisting}
        qemu-system-x8664 -hda /path/to/diskimage.img -m 1024 -boot d -usb -device usb-host,hostbus=1,hostaddr=2
    \end{lstlisting}
    Где \textit{hostbus} и \textit{hostaddr} — это идентификаторы USB-устройства, которое необходимо пробросить

    \subsection{Управление виртуальными машинами}
    
    \subsection{Снапшоты}
    Создание снапшота позволяет сохранить текущее состояние виртуальной машины для последующего восстановления.

    \subsection{Создание снапшота}
    \begin{lstlisting}
        qemu-img snapshot -c snapshotname /path/to/diskimage.img
    \end{lstlisting}

    \subsection{Восстановление снапшота}
    \begin{lstlisting}
        qemu-system-x8664 -hda /path/to/diskimage.img -m 1024 -boot d -loadvm snapshotname
    \end{lstlisting}

    \subsection{Преобразование форматов диска}
    QEMU позволяет преобразовывать виртуальные диски из одного формата в другой. Например, для преобразования raw-образа в qcow2:
    \begin{lstlisting}
        qemu-img convert -f raw -O qcow2 /path/to/original.img/path/to/converted.qcow2
    \end{lstlisting}

    \subsection{Продвинутые возможности}

    \subsection{Сетевые мосты}
    Для создания мостовой сети между хостом и виртуальными машинами можно использовать следующие команды:
    Создание мостового интерфейса:
    \begin{lstlisting}
        sudo ip link add name br0 type bridge
        sudo ip link set dev br0 up
        sudo ip addr add 192.168.1.1/24 dev br0
    \end{lstlisting}
    Настройка виртуальной машины для использования мостового интерфейса:
    \begin{lstlisting}
        qemu-system-x8664 -hda /path/to/diskimage.img -m 1024 -boot d -netdev tap,id=net0,ifname=tap0,script=no,downscript=no -device e1000,netdev=net0 -net bridge,br=br0
    \end{lstlisting}

    \subsection{Ускоренная графика}
    Чтобы улучшить производительность графики, QEMU может использовать виртуализованный графический адаптер:
    \begin{lstlisting}
        qemu-system-x8664 -hda /path/to/diskimage.img -m 1024 -boot d -vga qxl
    \end{lstlisting}

    \subsection{Специфические настройки для архитектур}
    QEMU предоставляет возможности для настройки виртуальных машин, специфичные для различных архитектур процессоров. Например, для ARM:
    \begin{lstlisting}
        qemu-system-arm -M versatilepb -cpu cortex-a8 -m 512 -kernel /path/to/kernel -initrd /path/to/initrd -append "root=/dev/ram"
    \end{lstlisting}

    \subsection{Инструменты и утилиты}

    \begin{itemize}
        \item \textbf{qemu-img} \\
            \textit{qemu-img} — это утилита для создания, преобразования и управления файлами образов жестких дисков. \\
            Создание нового образа диска: \\
            \begin{lstlisting}
                qemu-img create -f qcow2 /path/to/diskimage.qcow2 20G
            \end{lstlisting}
            Просмотр информации об образе диска: \\
            \begin{lstlisting}
                qemu-img info /path/to/diskimage.qcow2
            \end{lstlisting}
        \item \textbf{qemu-nbd} \\
            \textit{qemu-nbd}— это утилита для подключения образов дисков как сетевых блочных устройств. \\
            Подключение образа диска: \\
            \begin{lstlisting}
                sudo qemu-nbd -c /dev/nbd0 /path/to/diskimage.qcow2
            \end{lstlisting}
            Отключение образа диска: \\
            \begin{lstlisting}
                sudo qemu-nbd -d /dev/nbd0
            \end{lstlisting}
    \end{itemize}

    QEMU — это мощный инструмент для эмуляции и виртуализации, предлагающий широкий спектр возможностей для работы с различными архитектурами и конфигурациями. Благодаря своей гибкости и поддержке множества функций, QEMU является популярным выбором как для разработчиков, так и для системных администраторов.

\endinput