\chapter*{Реферат}

    Итоговая аттестационная работа состоит из 158 страниц, 15 рисунков, 2 таблиц, 0 использованных источников, 0 приложений.

    Итоговая аттестационная работа выполнена в формате IT-проекта на тему «Портирование нейронных сетей компьютерного зрения на бортовые вычислительные комплексы БПЛА».
    
    Дипломный проект основан на использовании нейросетевых технологий, позволяющие точно и быстро анализировать большое количество данных.
    
    В результате была разработана методика портирования модели, способной решать задачу повышения автономности за счет детекции объектов при съемке с БПЛА, используя малогабаритные вычислители, предназначенные для расположения на борту беспилотного летательного аппарата.
    
    Для достижения поставленной цели были проведены следующие действия: сбор данных, разработка и обучение нейросетевой модели, её оценка и улучшение, запуск нейросети на виртуальной машине, написание тестов, анализ и адаптация программы для работы на реальном вычислителе. Основное содержание работы состояло в разработке,  обучении нейросети и создании метода ее портирования. 
    
    Основным результатом работы, полученным в процессе разработки, является предварительно обученная нейросеть, оптимизированная для распознавания чрезвычайных ситуаций в реальном времени с помощью портативного вычислителя на беспилотнике, написанная на Python с использованием библиотек TensorFlow, PyTorch, NumPy, Pandas, CUDA и др.
    
    Данные результаты разработки предназначены для работы в военной отрасли и службах спасения, которые занимаются поиском людей после/во время природных катаклизмов, в сложных условиях. Разработку можно использовать во множестве организаций в разных отраслях. Решение можно модифицировать под желания компании с помощью интеграции модели с сущствующими системами.
    
    Применение этого решения позволяет организациям портировать нейронные модели на различные вычислительные комплексы БПЛА и повышать их автономность при работе в сложных условиях.

\endinput