\chapter*{Введение}
\addcontentsline{toc}{chapter}{Введение}
\label{ch:intro}

    В последние годы наблюдается значительный рост использования беспилотных летательных аппаратов (БПЛА) в различных областях, таких как сельское хозяйство, экология, безопасность, промышленность и даже доставка товаров. Применение БПЛА существенно расширяет возможности мониторинга, сбор данных и выполнения различных задач, которые были бы трудными или даже невозможными для человека. В этом контексте компьютерное зрение играет ключевую роль, предоставляя способность анализировать и интерпретировать изображения и видео в реальном времени.

    Можно выделить основные потребности в решении поставленной задачи:
    \begin{enumerate}
        \item Повышение автономности БПЛА \\
        Одной из основных преимуществ использования нейронных сетей для компьютерного зрения на борту БПЛА является заметное повышение уровня автономности. Традиционные системы требуют передачи данных на наземные станции для дальнейшей обработки, что не только замедляет процесс, но и ограничивает радиус действия БПЛА. Перенос вычислительных способностей на борт позволяет осуществлять анализ и принятие решений в реальном времени, что чрезвычайно важно для задач, требующих быстрой реакции, таких как обнаружение препятствий, навигация в сложных условиях и автономная посадка.
        \item Применение в различных отраслях
            \begin{itemize}
                \item Сельское хозяйство: \\
                Использование БПЛА для мониторинга состояния полей, выявления участков с заболеваниями растений, оценки эффективности сельскохозяйственных методов и прогнозирования урожайности становится все более популярным. Нейронные сети способны анализировать изображения полей с целью выявления аномалий или подсчета всходов, что помогает фермерам принимать информированные решения.
                \item Экология и защита окружающей среды: \\
                Нейронные сети могут быть использованы для мониторинга состояния лесов, отслеживания популяций животных, выявления незаконных свалок и других экологических нарушений. Автономное выполнение этих задач с помощью БПЛА позволяет регулярно обновлять данные и оперативно реагировать на изменения.
                \item Безопасность: \\
                Для служб безопасности и министерств обороны важно иметь возможность оперативного мониторинга местности, проверки объектов и проведения разведывательных операций. Компьютерное зрение на базе нейронных сетей позволяет эффективно идентифицировать потенциальные угрозы и реагировать на них немедленно.
                \item Промышленность: \\
                В промышленности БПЛА могут использоваться для инспекции инфраструктуры, такой как энергетические сети, трубопроводные системы и мосты. Автоматическое выявление дефектов или повреждений может значительно снизить риски и затраты на регулярные проверки.
            \end{itemize}
        \item Преодоление ограничений аппаратных ресурсов \\
        Прямое портирование нейронных сетей, разработанных для мощных серверов, на бортовые вычислительные комплексы БПЛА сталкивается с рядом проблем, таких как ограниченная вычислительная мощность, энергоэффективность и размеры устройств. Необходимы специализированные методики оптимизации и наладки, чтобы нейронные сети могли эффективно работать в условиях ограниченных ресурсов.
    \end{enumerate}

    Современные достижения в области информационных технологий открывают множество новых возможностей, которые могут существенно облегчить и ускорить процесс портирования нейронных сетей компьютерного зрения на бортовые вычислительные комплексы беспилотных летательных аппаратов (БПЛА). Рассмотрим наиболее значимые из них.

    \begin{enumerate}
        \item Усовершенствование аппаратного обеспечения \\
        Архитектуры специализированных процессоров: \\
        Современные процессоры специализированных для задач машинного обучения архитектур, такие как GPU (Graphics Processing Unit), TPU (Tensor Processing Unit) и FPGA (Field Programmable Gate Arrays), предоставляют огромные вычислительные мощности для обработки сложных нейронных сетей. Эти процессоры могут эффективно работать с параллельными вычислительными задачами, что делает их идеальными для реальных приложений компьютерного зрения на борту БПЛА.
        
        Edge AI процессоры: \\
        Появление специализированных процессоров, ориентированных на вычисления на периферии сети (edge computing), таких как NVIDIA Jetson, Movidius Myriad, Google Edge TPU, позволяет размещать мощные алгоритмы машинного обучения непосредственно на устройствах с ограниченными ресурсами. Они оптимизированы для низкого энергопотребления, что критически важно для автономных БПЛА.
        \item Развитие программного обеспечения и инструментов \\
        Фреймворки для глубокого обучения: \\
        Программные фреймворки, такие как TensorFlow, PyTorch, Caffe и ONNX, значительно упростили процесс разработки нейронных сетей. Они предоставляют богатые библиотеки и инструменты для проектирования, обучения и оптимизации моделей. Более того, они поддерживают переносимость моделей на различные аппаратные платформы, что облегчает портирование моделей на бортовые вычислительные комплексы.
        
        Инструменты для оптимизации моделей: \\
        Существуют специализированные инструменты, такие как TensorRT для NVIDIA GPU, OpenVINO для Intel устройств и TFLite для мобильных и встраиваемых систем, которые позволяют существенно оптимизировать модели машинного обучения для работы на устройствах с ограниченными ресурсами. Эти инструменты включают методы квантования, праунинг и другие подходы, снижающие интенсивность вычислений и энергопотребление.
        
        Системы синтетического моделирования и симуляции: \\
        Современные среды симуляции, такие как Gazebo, Unreal Engine, Carla и другие, позволяют создать виртуальные тестовые полигоны для проверки алгоритмов БПЛА в различных сценариях. Это позволяет разработчикам протестировать и оптимизировать модели машинного обучения в безопасных и контролируемых условиях до их развертывания на реальных устройствах.
        \item Достижения в области алгоритмов и архитектур \\
        Эффективные архитектуры нейронных сетей: \\
        Современные исследования в области нейронных сетей привели к созданию новых, более эффективных архитектур, таких как MobileNets, EfficientNet, ShuffleNet и другие. Эти архитектуры разработаны с учетом ограничений мобильных и встраиваемых устройств, обеспечивая высокое качество распознавания при значительно меньших вычислительных затратах.
        
        Методы квантования и праунинга: \\
        Квантование нейронных сетей, которое включает преобразование весов и активностей модели в формат с меньшей точностью (например, 8-битная целая арифметика), и праунинг (удаление ненужных параметров) помогают существенно снизить размеры моделей и повысить их производительность на ограниченных устройствах. Эти методы позволяют экономить энергию и ускорять вычисления без значительной потери точности.
        \item Интеллектуальные системы управления и оптимизации \\
        Автоматизация машинного обучения (AutoML): \\
        Технологии AutoML, такие как Google AutoML и AutoKeras, облегчают процесс проектирования и оптимизации нейронных сетей, автоматически подбирая оптимальную архитектуру и гиперпараметры. Это особенно полезно для разработчиков, которые хотят быстро адаптировать свои модели для работы на различных аппаратных платформах без глубоких знаний в области машинного обучения.
        
        Интерфейсы программирования и API: \\
        Платформы облачных вычислений, такие как AWS IoT Greengrass, Azure IoT Edge и Google Cloud IoT, предлагают мощные инструменты и интерфейсы (API) для интеграции нейронных сетей с периферийными устройствами. Эти решения позволяют реализацию гибридных вычислительных схем, где часть обработки может быть выполнена на периферии, а часть в облаке, предоставляя баланс между производительностью и потреблением ресурсов.
    \end{enumerate}

    Решение задачи портирования нейронных сетей компьютерного зрения на бортовые вычислительные комплексы беспилотных летательных аппаратов (БПЛА) может существенно повлиять на различные аспекты деятельности в производстве, бизнесе, отраслях и обществе в целом. Давайте рассмотрим ключевые преимущества и возможные применения этого решения, а также тех, кто может быть заинтересован в его реализации.
    
    Влияние на производство и бизнес:
    \begin{enumerate}
        \item Повышение эффективности и точности: \\
        Компьютерное зрение позволяет автоматизировать многие процессы, которые ранее требовали человеческого вмешательства. В производстве это может включать:
        \begin{itemize}
            \item Контроль качества: автоматический визуальный контроль продукции на производственных линиях.
            \item Мониторинг оборудования: своевременное обнаружение дефектов или неисправностей в оборудовании.
        \end{itemize}
        Компьютерное зрение на БПЛА позволяет собирать данные с труднодоступных или опасных мест, что увеличивает точность и оперативность контроля.
        \item Уменьшение затрат: \\
        Автоматизация и улучшение мониторинга с помощью компьютерного зрения может привести к значительному снижению затрат на рабочую силу и обслуживание. Это также может снизить количество ошибок и, соответственно, затрат на их исправление.
        \item Улучшение безопасности: \\
        БПЛА с функциями компьютерного зрения могут проводить мониторинг опасных зон, таких как шахты, нефтеперерабатывающие заводы или высоковольтные линии электропередач, без риска для человеческих жизней.
    \end{enumerate}

    \section*{Влияние на различные отрасли}
    \begin{enumerate}
        \item Сельское хозяйство:
        Использование БПЛА с компьютерным зрением позволяет:
        \begin{itemize}
            \item Мониторинг урожая: раннее обнаружение вредителей или заболеваний.
            \item Оптимизация полива и удобрений: анализ состояния почвы с высокой точностью.
            \item Картирование полей: точное измерение площади и состояния посевов.
        \end{itemize}
        
        \item Транспорт и логистика:
        БПЛА с компьютерным зрением могут обеспечить:
        \begin{itemize}
            \item Мониторинг инфраструктуры: инспекция дорожных покрытий, мостов и железных дорог.
            \item Складская автоматизация: инвентаризация и отслеживание товаров в реальном времени.
            \item Автономные доставочные системы: эффективная и безопасная доставка товаров.
        \end{itemize}
        
        \item Строительство и недвижимость:
        Компьютерное зрение на БПЛА может использоваться для:
        \begin{itemize}
            \item Инспекции строительных объектов: автоматическое отслеживание прогресса строительства.
            \item Создание 3D-моделей: создание цифровых двойников зданий и инфраструктуры.
            \item Оценка состояния объектов: обнаружение структурных дефектов или износа.
        \end{itemize}
    \end{enumerate}
    
    \begin{enumerate}
        \item Улучшение качества жизни:
        БПЛА с компьютерным зрением могут участвовать в спасательных операциях, мониторинге экологии, предсказании стихийных бедствий, что позволяет быстрее реагировать на чрезвычайные ситуации и минимизировать ущерб.
        
        \item Развитие умных городов:
        Компьютерное зрение на БПЛА может способствовать развитию умных городов, предоставляя данные для улучшения транспортных систем, мониторинга загрязнения воздуха, управления энергопотреблением и других аспектов городской инфраструктуры.
    \end{enumerate}
    
    \section*{Заказчики и потребители решения}
    
    \begin{enumerate}
        \item Промышленные компании:
        Производственные и инженерные предприятия заинтересованы в:
        \begin{itemize}
            \item Увеличении точности и эффективности своих процессов.
            \item Снижении затрат на мониторинг и обслуживание оборудования.
        \end{itemize}
        
        \item Аграрные компании:
        Сельскохозяйственные предприятия могут использовать БПЛА с компьютерным зрением для:
        \begin{itemize}
            \item Повышения урожайности.
            \item Оптимизации использования ресурсов.
        \end{itemize}
        
        \item Логистические и транспортные компании:
        Компании в сфере логистики и транспорта могут извлечь выгоду из:
        \begin{itemize}
            \item Автоматизации мониторинга инфраструктуры.
            \item Управления складами.
        \end{itemize}
        
        \item Строительные и девелоперские компании:
        Строительные фирмы могут более эффективно управлять своими проектами, используя возможности компьютерного зрения для:
        \begin{itemize}
            \item Мониторинга и оценки прогресса.
            \item Оценки состояния строительных объектов.
        \end{itemize}
        
        \item Государственные и муниципальные органы:
        Местные и федеральные органы власти могут использовать эти технологии для:
        \begin{itemize}
            \item Мониторинга экологической обстановки.
            \item Обеспечения безопасности.
            \item Управления городской инфраструктурой.
        \end{itemize}
        
        \item Спасательные и экологические организации:
        Организации, занимающиеся спасательными операциями и защитой окружающей среды, могут использовать БПЛА с компьютерным зрением для:
        \begin{itemize}
            \item Быстрого реагирования на чрезвычайные ситуации.
            \item Мониторинга экологического состояния.
        \end{itemize}
    \end{enumerate}
    
    \section*{Насколько это решение необходимо?}
    
    Необходимость в таких решениях продиктована несколькими ключевыми факторами:
    \begin{enumerate}
        \item Рост потребностей в автоматизации:
        Во многих отраслях растет потребность в автоматизации процессов для повышения эффективности и точности. Компьютерное зрение на БПЛА предоставляет возможности для удовлетворения этих потребностей.
        
        \item Безопасность и здоровье работников:
        Использование БПЛА для выполнения опасных задач снижает риск для человеческой жизни и здоровья, что делает такие технологии критически важными в ряде случаев.
        
        \item Экономия ресурсов и времени:
        Компьютерное зрение позволяет сократить время и затраты на выполнение задач, что особенно важно для бизнесов и организаций, работающих в условиях высокой конкуренции и жестких бюджетов.
        
        \item Увеличение требований к качеству:
        С ростом ожиданий потребителей и ужесточением стандартов качества растет потребность в более точных и быстрых системах контроля, что делает технологии компьютерного зрения на БПЛА особенно актуальными.
    \end{enumerate}

    В итоге, решение задач по портированию нейронных сетей компьютерного зрения на бортовые вычислительные комплексы БПЛА не только приносит значительные преимущества для различных индустрий и общественных секторов, но и становится все более необходимым в условиях растущих требований к автоматизации, точности, безопасности и эффективности.

    \section*{Цель задачи}
    
    Целью задачи является разработка методики портирования модели, способной решать задачу повышения автономности за счет детекции объектов при съёмке с БПЛА, используя малогабаритные вычислители, предназначенные для установки на борту беспилотного летательного аппарата.
    
    Образом результата является предварительно обученная нейросеть, оптимизированная для распознавания чрезвычайных ситуаций в реальном времени с помощью портативного вычислителя на беспилотнике.
    
    Для достижения поставленного образа результата необходимо выполнить следующие задачи:
    \begin{enumerate}
        \item Подготовка хранилища данных:
        \begin{itemize}
            \item Сбор и аннотирование данных для обучения модели.
            \item Организация эффективного хранения и доступа к данным.
        \end{itemize}
        
        \item Разработка и обучение нейросети:
        \begin{itemize}
            \item Проектирование архитектуры нейросети, подходящей для задачи детекции объектов.
            \item Обучение нейросети с использованием подготовленных данных.
        \end{itemize}
        
        \item Верификация результатов работы системы:
        \begin{itemize}
            \item Оценка точности и производительности обученной модели на тестовых данных.
            \item Проведение тестирования модели в условиях, приближенных к реальным.
        \end{itemize}
        
        \item Адаптация программы для реального вычислителя:
        \begin{itemize}
            \item Оптимизация модели для работы на малогабаритном вычислителе.
            \item Интеграция модели с программным обеспечением БПЛА.
            \item Тестирование и отладка системы на реальном оборудовании.
        \end{itemize}
    \end{enumerate}

\endinput